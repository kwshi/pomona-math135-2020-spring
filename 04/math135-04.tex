\documentclass{../math135}

\title{Homework 4}
\author{}
\date{Tuesday 2/18}

\begin{document}

\begin{exercise}
	Does there exist an analytic function
  \(f \colon \DD \to \DD \setminus \set 0\) which is surjective and
  such that the preimage \(f^{-1}(\set w)\) of each point
  \(w \in \DD \setminus \set 0\) is infinite?

  \begin{solution}

  \end{solution}
\end{exercise}

\begin{exercise}
	Let \(\Omega\) be a region in \(\CC\) and let
  \(f_n \colon \Omega\to\CC\) be a sequence of functions that
  converges pointwise on \(\Omega\) to a function
  \(f \colon \Omega\to\CC\).  Consider the following definitions.
	\begin{enumerate}
  \item \(f_n\) converges to \(f\) \emph{locally uniformly} on
    \(\Omega\) if for each \(z \in \Omega\) there exists
    \(\delta > 0\) such that \(f_n\) converges uniformly to \(f\) on
    the open disk \(B(z, \delta) \subseteq \Omega\).
    \label{itm:local-uniform}
  \item \(f_n\) converges to \(f\) \emph{uniformly on compact sets} in
    \(\Omega\) if for each compact subset \(K \subset \Omega\), the
    sequence \(f_n\) converges uniformly to \(f\) on \(K\).
    \label{itm:uniform-compact}
	\end{enumerate}
	Prove that these two concepts are equivalent.\footnote{These
    concepts are important mainly because one rarely obtains uniform
    convergence on \(\Omega\) (bad things may happen on or near
    \(\partial\Omega\)).  However, one often obtains locally uniform
    convergence.  For instance, a power series with radius of
    convergence \(R > 0\) typically does not converge uniformly on
    \(\abs*{z-z_0} < R\), but it does converge locally uniformly on
    \(\abs*{z-z_0} < R\).

		Local uniform convergence is enough for most purposes in complex
    function theory.  Many upcoming theorems depend upon integrating
    over closed curves in \(\Omega\).  Since closed curves are
    \emph{compact}, locally uniform convergence on \(\Omega\) gives
    (via the equivalence \ref{itm:local-uniform} \(\iff\)
    \ref{itm:uniform-compact}) uniform convergence on each closed
    curve.  This is enough to justify certain operations, like
    swapping limits with contour integrals.}

  \begin{solution}

  \end{solution}

\end{exercise}

\begin{exercise}
	Let \(f(z) = \frac{z}{1-z-z^2}\).
	\begin{problems}
  \item Assume that \(f(z) = \sum_{n=0}^{\infty} a_n z^n\) on
    \(\abs z < R\) for some \(R > 0\).  Prove that \(a_n = f_n\) where
    \(f_n\) is the \(n\)th \emph{Fibonacci number}
    \(0,1,1,2,3,5,8,13,\ldots\).\footnote{\(f_0 = 0\), \(f_1 = 1\),
      and \(f_n = f_{n-1}+ f_{n-2}\) for \(n \geq 2\).}

    \begin{solution}

    \end{solution}

  \item Perform a partial fraction expansion and use the geometric
    series summation formula to derive\footnote{Hint: Avoid
      manipulating expressions involving \(\sqrt 5\)s if you can.  For
      instance, no \(\sqrt 5\)s appear in my solution until the
      next-to-last line.  You can assume that the coefficients in a
      power series expansion are unique. } \emph{Binet's Formula}
    \begin{equation}
      f_n = \frac{1}{\sqrt 5} \prn*{
        \prn*{\frac{1 + \sqrt 5}{2}}^n
        - \prn*{\frac{1 - \sqrt 5}{2}}^n
      }
      \label{eq:binet}
    \end{equation}
    for the \(n\)th Fibonacci number.  Note that \eqref{eq:binet} is
    \emph{explicit}, in the sense that \(f_n\) can be computed
    directly, without any knowledge of previous Fibonacci numbers.
    \comment{Since \(\frac{1-\sqrt 5}{2} \approx -0.618\), the second
      term in \eqref{eq:binet} tends to zero exponentially fast.
      Therefore \(\frac{1}{\sqrt 5} \prn*{\frac{1 + \sqrt 5}{2}}^n\)
      is an excellent approximation to the \(n\)th Fibonacci number.
      It yields the remarkable approximations \(55.00363612\) and
      \(6765.000030\) for \(f_{10} = 55\) and \(f_{20} = 6765\).}

    \begin{solution}

    \end{solution}

	\end{problems}
\end{exercise}

\begin{exercise}
	Certain power series can be \emph{analytically continued} beyond
  their disk of convergence.  For instance, the geometric series
  \(\sum_{n=0}^\infty z^n\) converges on the open unit disk \(\DD\).
  However, it is only a ``small part'' of the analytic function
  \(\frac{1}{1-z}\), which is defined on all of
  \(\CC \setminus \set 1\).  Consider the function
  \[
		f(z) = \sum_{n=0}^\infty z^{2^n}.
  \]

	\begin{problems}
  \item Show that the series defining \(f(z)\) has radius of
    convergence \(R = 1\).

    \begin{solution}

    \end{solution}

  \item Show that \(\lim_{z\to\zeta} \abs*{f(z)}\) diverges if
    \(\zeta\) is a \(2^m\)th root of unity.  Since the set of
    \(2^m\)th roots of unity is dense in the unit circle \(\TT\),
    \(f\) cannot be analytically continued across any arc of \(\TT\).
    In other words, the unit circle is the \emph{natural boundary} of
    \(f\).\footnote{Hint: Use
      \(f(z) = \prn*{z + z^2 + z^4 + \cdots + z^{2^{m-1}}} +f
      \prn*{z^{2^m}}\) and \(\lim_{r \to 1^-} f(r) = \infty\).}

    \begin{solution}

    \end{solution}

	\end{problems}
\end{exercise}

\end{document}
