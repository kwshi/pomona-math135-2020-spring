\documentclass{../math135}

\title{Homework 3}
\author{}
\date{Tuesday 2/11}

\begin{document}

\begin{exercise}
	Suppose that \(u \colon \Omega\to\RR\) is harmonic and twice
  continuously differentiable on the open disk
  \(\Omega = B(z_0, \delta)\) of radius \(\delta > 0\) centered at
  \(z_0 = (x_0, y_0)\).  Prove that \(u\) has a harmonic conjugate
  \(v \colon \Omega\to\RR\) on \(\Omega\).\footnote{Hint: Attempt to
    derive a formula for \(v(x, y)\) using the Cauchy--Riemann
    Equations and the Fundamental Theorem of Calculus.  Why is the
    shape of \(\Omega\) important?}
\end{exercise}


\begin{exercise}
	Let \(g \colon \Omega\to\CC\) be analytic and \(u \colon U\to\RR\)
  be harmonic.\footnote{The domains of analytic or harmonic functions
    are assumed to be \emph{regions} (nonempty, connected open sets).}
  If \(g(\Omega) \subseteq U\), does it follows that
  \(u \circ g \colon \Omega\to\RR\) is harmonic?  Justify your
  answer.\footnote{Hint: Use exercise 1.  Also recall that the inverse
    image of an open set under a continuous function is open.}%
\end{exercise}

\begin{exercise}
  Let \(\DD\) denote the open unit disk and let \(\DD^-\) denote its
  closure.
  \begin{enumerate}
	\item Suppose that \(f\) is analytic on \(\DD\) and continuous on
    \(\DD^-\).  If \(f\) is real-valued on the unit circle \(\TT\),
    must \(f\) be constant?\footnote{Hint: First note that \(\DD^-\)
      is compact.  Use the Maximum Principle for Harmonic Functions.}%


	\item Suppose that \(f\) is analytic on \(\DD\) and continuous on
    \(\DD^- \setminus \set 1\).  If \(f\) is real-valued on
    \(\TT \setminus \set 1\), must \(f\) be constant?


  \end{enumerate}
\end{exercise}



\begin{exercise}
	Find the M\"obius transformation
  \[
    f(z) = \frac{az+b}{cz+d}
  \]
  which satisfies \(f(1) = 1\), \(f(-1) = -1\), and
  \(f(i) = \infty\).\footnote{Since M\"obius transformations preserve
    extended circles, it follows that \(f\) maps \(\DD\) onto the
    \emph{upper half-plane} \(\UU = \{z \in \CC: \Im z > 0\}\).  We
    can deduce this from the fact that the domain \(\DD\) is on our
    left as we traverse the unit circle counterclockwise from \(1\) to
    \(i\) to \(-1\).  Since analytic functions are
    orientation-preserving, the image of \(f(\DD)\) must lie on our
    left as we traverse the image of the unit circle (i.e., the
    extended real line) from \(1\) to \(+\infty\) (which is also
    \(-\infty\)) to \(-1\).  Thus, the image of \(f(\DD)\) must be the
    upper half-plane, as opposed to the lower half-plane.  Another way
    to see this is to evaluate \(f(0)\) once you find the explicit
    formula for \(f(z)\).  You will see that \(f(0)\) lies in the
    upper half-plane, and hence \(f\) must map \(\DD\) onto \(\UU\).}
  Is \(f\) the only analytic function on \(\CC \setminus \set i\)
  which satisfies \(f(1) = 1\), \(f(-1) = -1\), and \(f(i) = \infty\)?
\end{exercise}


\begin{exercise}
	Fix \(a > 0\).  Describe the image of the infinite horizontal strip
  \(S= \set*{z \in \CC : 0 < \Im z < \frac{1}{2a}}\) under the mapping
  \(f(z) = \frac 1 z\).
\end{exercise}


\begin{exercise}
	Find a rational function \(f \colon \DD\to\CC\) that is surjective.
\end{exercise}

\begin{exercise}
	Find an analytic bijection \(f \colon \Omega\to\DD\) from the
  quarter disk
  \[
		\Omega = \set*{ z \in \CC
      : \abs z < 1, \, \Re z > 0, \, \Im z > 0 }
  \]
	onto \(\DD\).  Justify your answer.\footnote{Do not use fancy
    theorems we have not yet covered.  Also note that the answer is
    \textbf{not} \(f(z)=z^4\).  Why?}
\end{exercise}

\end{document}
