\documentclass{../math135}

\title{Homework 9}
\author{}
\date{Tuesday 4/7}

\begin{document}

\begin{exercise}
	Suppose that \(f\) is entire and \(\abs* {\frac {f(z)} {z^n}} \le
  M\) for \(\abs z > R\).  Prove that \(f\) is a polynomial of degree
  \(\leq n\).

  \begin{solution}

  \end{solution}
\end{exercise}

\begin{exercise}
	Suppose \(f \colon \DD \to \CC\) is analytic and
  \[
		\lim_{\substack{z \to 1 \\ z \in \DD}}
    \frac{\abs{f(z)}^2}{\abs{z-1}} = 1.
  \]
	Prove that \(f\) cannot be analytically continued to a neighborhood
  of \(z_0 = 1\).

  \begin{solution}

  \end{solution}
\end{exercise}

\begin{exercise}
  \comment{A \emph{fixed point} of \(f\) is a point \(\xi\) such that
    \(f(\xi) = \xi\).}  Suppose that \(f\) is analytic on
  \(\CC \setminus \set 0\), \(f(1) = 2\), and \(f(2) = 1\).  Is it
  possible for \(f\) to have infinitely many fixed points in the
  annular region \(1 < \abs z < 2\)?

  \begin{solution}

  \end{solution}
\end{exercise}

\begin{exercise}
	Let \(p(z)\) be a nonconstant polynomial and let \(\gamma\) be a
  simple closed curve in \(\CC\) that does not pass through any zero
  of \(p(z)\).  Give a general rule for evaluating
  \[
		\frac{1}{2 \pi i} \int_\gamma
    \frac{p'(\zeta)}{p(\zeta)} \dif \zeta.
  \]

  \begin{solution}

  \end{solution}
\end{exercise}

\begin{exercise}
	Do there exist \(f, g\), analytic in a neighborhood of \(z = 0\),
  such that
  \begin{problems}
  \item \(f \prn*{\frac 1 n} = f \prn*{-\frac 1 n} = \frac{1}{n^2}\)
    for \(n = 1, 2, \dots\)?
  \item \(g \prn*{\frac 1 n} = g \prn*{-\frac 1 n} = \frac{1}{n^3}\)
    for \(n = 1, 2, \dots\)?
  \end{problems}
	Justify your answer.

  \begin{solution}

  \end{solution}
\end{exercise}


\begin{exercise}
	If \(f \colon \DD \to \DD\) is analytic, prove that
  \[
		\abs*{\frac{f(z) - f(w)}{1 - \overline{f(w)} f(z)}} \le
    \abs*{\frac{z-w}{1 - \bar w z}}
  \]
	hold for all \(z, w \in \DD\).\footnote{It turns out that
    \(\rho(z, w) = \abs*{\frac{z-w}{1 - \bar w z}}\) defines a metric
    on \(\DD\), called the \emph{pseudohyperbolic metric}, which can
    be used to construct the \emph{hyperbolic} (of \emph{Poincar\'e})
    metric \(\psi(z, w) = \log \frac{1 + \rho(z,w)}{1 - \rho(z,w)}\)
    on \(\DD\).  When equipped with the Poincar\'e metric, the unit
    disk \(\DD\) becomes a model of the \emph{hyperbolic plane}.  In
    other words, one can concretely work on hyperbolic geometry by
    working on \(\DD\).  This problem is essentially asking you to
    prove that an analytic function \(f \colon \DD \to \DD\) is a
    \emph{contraction} with respect to the pseudohyperbolic (and hence
    the Poincar\'e) metric.}
\end{exercise}



\end{document}