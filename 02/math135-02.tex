\documentclass{../math135}

\title{Homework 2}
\author{}
\date{Tuesday 2/4}

\begin{document}

\begin{exercise}
	Let \(\abs a = \abs b = \abs c \ne 0\) and \(a + b + c = 0\).  Prove
  that \(a, b, c\) form the vertices of an equilateral triangle
  inscribed in a circle centered at the origin.

  \begin{solution}
  \end{solution}

\end{exercise}

\begin{exercise}
	Prove that \(\lim_{z\to \infty} p(z) = \infty\) holds for any
  nonconstant polynomial.\footnote{Hint: Suppose that
    \(p(z) = a_n z^n + a_{n-1} z^{n-1} + \cdots + a_1 z + a_0\) with
    \(a_n \ne 0\).  Prove that for any \(M >0\) there exists \(R>0\)
    such that \(\abs{p(z)} > M\) whenever \(\abs z > R\).  Use the
    \emph{reverse triangle inequality}
    \(\abs*{\abs a - \abs b} \le \abs{a-b}\) to show that the
    \(a_n z^n\) term dominates.}

  \begin{solution}
  \end{solution}

\end{exercise}

\begin{exercise}
	Show that the functions \(u(x, y) = e^x \cos y\) and
  \(v(x, y) = e^x \sin y\) are harmonic conjugates on \(\CC\).  In
  particular, this shows that the function
  \(\exp(z) = e^x\cos y + i e^x \sin y\) (where \(z=x+iy\)) defines an
  analytic function on \(\CC\).

  \begin{solution}
  \end{solution}

\end{exercise}

\begin{exercise}
	Let \(\Omega\) be a region in \(\CC\).  Using the Cauchy--Riemann
  equations, identify all analytic \(f \colon \Omega\to\CC\) which satisfy
  the following (consider each problem separately).
	\begin{problems}
  \item \(f(\Omega) \subseteq \RR\),

    \begin{solution}
    \end{solution}

  \item \(\arg f\) is constant on \(\Omega\),

    \begin{solution}
    \end{solution}

  \item \(\overline{f}\) is analytic on \(\Omega\),

    \begin{solution}
    \end{solution}

  \item \(\abs f\) is constant on \(\Omega\)

    \begin{solution}
    \end{solution}

  \item \(f(x,y) = u(x) + i v(y)\), in which \(u,v\) are real-valued.

    \begin{solution}
    \end{solution}

	\end{problems}
\end{exercise}

\begin{exercise}
	The \emph{Poisson kernel} is the family of functions \(P_r(t)\) defined by
	\begin{equation*}
		P_r(t) = \frac{1 - r^2}{1 - 2r \cos t + r^2},
	\end{equation*}
	where \(0 \le r < 1\) and \(-\pi \le t \le \pi\) (see Figure
  \ref{FigurePoisson}).  \footnote{The Poisson kernel arises in a
    number of important contexts (e.g., solving the steady-state heat
    equation on \(\DD\)).}
	  \begin{figure}[h]
	    \begin{center}
	      %\includegraphics[width=3in]{Graphs/Poisson}
        \tikzset{
          declare function={
            poisson(\r,\t)=(1-\r*\r)/(1-2*\r*cos(deg(\t))+\r*\r);
          },
          poisson/.style={
            samples=257,
            domain=-pi:pi,
          },
        }
        \begin{tikzpicture}
        	\begin{axis}[
            width=4in,
            height=5in/2,
            axis y line=center,
            axis x line=bottom,
            ylabel=\(P_r(t)\),
            ylabel style=above,
            xlabel=\(t\),
            xlabel style=right,
            ymax=10,
            xtick={-pi,0,pi},
            xticklabels={\(-\pi\),\(0\),\(\pi\)},
            ]
            \addplot [poisson, draw=blue] {poisson(.2,x)};
            \addplot [poisson, draw=purple] {poisson(.5,x)};
            \addplot [poisson, draw=olive] {poisson(.8,x)};
          \end{axis}
        \end{tikzpicture}
	      \caption{\footnotesize The Poisson kernel \(P_r(t)\) for
          \(r = 0.2, \,0.5, \,0.8\).  As \(r\to 1^-\), the graphs
          spike sharply at \(t = 0\) while tending rapidly to zero for
          \(t\) away from \(0\).  Intuitively, the functions
          \(P_r(t)\) ``approximate a Dirac \(\delta\)-function at
          \(t = 0\)'' as \(r \to 1^-\).  This notion can be made more
          precise using measure theory and functional analysis (Math
          137--8).}
	      \label{FigurePoisson}
	    \end{center}
	  \end{figure}

	\begin{problems}
  \item Show that \(P_r(t) = \Re f(re^{it})\) where
    \(f(z) = \frac{1+z}{1-z}\).

    \begin{solution}
    \end{solution}

  \item Show that \(P_r(t)\) is harmonic in \(\DD\).\footnote{Hint:
      \(P_r(t)\) is given as a function of \(r,t\), rather than
      \(x,y\).  Verifying directly that \(P_r(t)\) satisfies the
      Laplace equation is hard since it requires working with the
      Laplace equation in polar coordinates.  Find an easier way.}

    \begin{solution}
    \end{solution}

	\end{problems}
\end{exercise}

\end{document}