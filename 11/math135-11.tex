\documentclass {../math135}

\title {Homework 11}
\date {Tuesday 4/21}
\author {}

\begin {document}

\begin {exercise}
	Suppose \(f\) and \(g\) are entire functions such that
  \(\abs {f(z)} \le \abs {g(z)}\) for all \(z \in \CC\).  Show that
  there exists a constant \(\lambda\) satisfying
  \(\abs \lambda \le 1\) such that \(f = \lambda g\).

  \begin {solution}

  \end {solution}

\end {exercise}

\begin {exercise}
	Let \(\UU = \set {z \in \CC : \Im z > 0}\).  Find a formula
  describing all analytic bijections \(f \colon \UU \to \CC\) (you may
  assume that \(f'(z) \ne 0\) for all \(z \in \UU\)).

  \begin {solution}

  \end {solution}

\end {exercise}

\begin {exercise}
  \begin {problems}
  \item Find an example of a complex-valued function
    \(f \colon \Omega \to \CC\) such that \(f^2\) is analytic on
    \(\Omega\) but such that \(f\) is not itself analytic.

    \begin {solution}

    \end {solution}

  \item What if we assume that both \(f^2\) and \(f^3\) are analytic
    on \(\Omega\)?

    \begin {solution}

    \end {solution}

  \end {problems}
\end {exercise}

\begin {exercise}
	Compute
  \(\frac 1 {2 \pi i} \int_{\abs {z-1} = 3} \frac {e^z - 1} {z \sin z}
  \dif z\).

  \begin {solution}

  \end {solution}

\end {exercise}

\begin {exercise}
	State and prove a generalization of the argument principle for
  functions having zeros \(z_1, z_2, \dots, z_n\) and poles
  \(w_1, w_2, \dots, w_m\) (repeated according to multiplicity) inside
  of \(\gamma\).

  \begin {solution}

  \end {solution}

\end {exercise}

\end {document}