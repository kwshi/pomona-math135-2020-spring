\documentclass{../math135}

\title{Homework 1}
\author{}
\date{Tuesday 1/28}

\begin{document}
\begin{exercise}
	Evaluate the following expressions.  Write your answer in the form
  \(a + bi\) where \(a\) and \(b\) are real.
	\begin{enumerate}
  \item \((1+i)(2+3i)\)

    \begin{solution}
    \end{solution}

  \item \(\dfrac{2+3i}{1+i}\)

    \begin{solution}
    \end{solution}

  \item \(\left(\dfrac{(2+i)^2}{4-3i}\right)^2\)

    \begin{solution}
    \end{solution}

  \item \(e^{i\alpha} \overline{e^{i\beta}}\), where \(\alpha\) and
    \(\beta\) are real.

    \begin{solution}
    \end{solution}

  \item \(\overline{(1-i)\overline{(2+2i)}}\)

    \begin{solution}
    \end{solution}

  \item \(\abs{2-i}^3\)

    \begin{solution}
    \end{solution}

	\end{enumerate}
\end{exercise}

\begin{exercise}
	Write \(z = \sqrt{3}-i\) in polar form \(z = re^{i\theta}\).

  \begin{solution}
  \end{solution}

\end{exercise}

\begin{exercise}
	What are the third roots of \(z=8i?\)

  \begin{solution}
  \end{solution}

\end{exercise}

\begin{exercise}\label{ExerciseInscribed}
	Let \(a,b,c\) be distinct complex numbers.  Prove that \(a,b,c\) are
  the vertices of an equilateral triangle if and only
  if\footnote{Hint: Consider the relationship between \(b-a\),
    \(c-b\), and \(a-c\).}
	\begin{equation}\label{eq-TriangleEquation}
		a^2 + b^2 + c^2 = ab+bc+ca.
	\end{equation}

  \begin{solution}
  \end{solution}

\end{exercise}

\begin{exercise}
	Let \(\DD = \{ z\in \CC : \abs z < 1\}\) denote the open unit disk,
  \(\DD^- = \{ z \in \CC: \abs z \leq 1\}\) the closed unit disk, and
  \(\TT = \{ z \in \CC : \abs z =1\}\) the unit circle.\footnote{The
    symbol \(\TT\) is frequently used to denote the unit circle.  The
    letter \(\TT\) stands for \emph{torus} (i.e., the surface of a
    donut).  Why?  Topologically speaking, the Cartesian product
    \(\TT^2 = \TT \times \TT\) of two tori a torus.  The unit circle
    \(\TT\) is a ``one-dimensional torus,'' \(\TT^2\) is the usual
    ``two-dimensional torus,'' etc.}  Fix \(a \in \DD\), and consider
  the function
	\begin{equation}\label{eq-Blaschke}
		f(z) = \frac{a-z}{1 - \overline{a} z}.
	\end{equation}
	\begin{enumerate}
	\item Show that \(\abs*{f(z)} <1\) for \(z \in \DD\).

    \begin{solution}
    \end{solution}

	\item Show that \(\abs*{f(z)} = 1\) for \(z \in \TT\).\footnote{Hint:
      \(\abs*{f(z)}^2 = f(z) \overline{f(z)}\).}

    \begin{solution}
    \end{solution}

	\item Restricting the domain of \(f\) to \(\DD\), show that the
    \(f \colon \DD\to\DD\) is a \emph{bijection} (i.e., one-to-one and onto).
    Find a formula for the inverse function
    \(f^{-1} \colon \DD\to\DD\).

    \begin{solution}
    \end{solution}

	\end{enumerate}
\end{exercise}

\begin{exercise}
	Prove that\footnote{L'H\^opital's Rule shows that the identity
    remains valid even if \(x\) is a multiple of \(2\pi\).  The
    requested identity is of fundamental importance in the study of
    \emph{Fourier series}.  The function on the right-hand side of
    \eqref{eq-DKF} is called the \emph{Dirichlet kernel}.}
	\begin{equation}
		1 + 2 \sum_{n=1}^N \cos nx = \frac{ \sin[ (N + \frac{1}{2})x] }{ \sin  \tfrac{x}{2}}\label{eq-DKF}
	\end{equation}
	for \(x \in \RR\) and \(N = 1,2,\ldots\) by using Euler's formula
  and the finite geometric series formula\footnote{This formula is
    used in calculus to sum the geometric series.  Indeed, if
    \(\abs z < 1\), then \(z^n \to 0\) so that
    \(\sum_{n=0}^{\infty} z^n = \lim_{N\to\infty} \sum_{n=0}^{N-1} z^n
    = \lim_{N\to\infty}\frac{1 - z^N}{1-z} = \frac{1}{1-z}\).}
	\begin{equation*}
		1+ z + \cdots + z^{n-1} = \frac{1-z^n}{1-z}.
	\end{equation*}

  \begin{solution}
  \end{solution}

\end{exercise}
\end{document}