\documentclass{../math135}

\title{Homework 7}
\author{}
\date{Tuesday 3/10}

\tikzset{
  ax/.style={annotation, x=1in+2em, y=1in+2em},
  contour/.style={
    x=1in,
    y=1in,
    postaction=decorate,
    decoration={
      markings,
      mark=between positions 1in/8 and 1 step 1in/4 with {\arrow{>}},
    },
  },
  every node/.style={text height=3em/4},
  dot/.append style={text height=0em},
}

\begin{document}

\begin{exercise}
	Suppose that \(p(z)\) is a polynomial of degree \(n \geq 2\) with
  all of its zeros in \(\abs z < r\).  Evaluate
  \[
		\int_{\abs z=r} \frac{\dif z}{p(z)}.
  \]

  \begin{solution}

  \end{solution}
\end{exercise}


\begin{exercise}\label{exerciseRectangle}
	Evaluate the improper integral
  \[
		\int_0^\infty e^{-x^2} \cos 2bx \dif x
  \]
	for \(b \ge 0\) by integrating \(f(z) = e^{-z^2}\) over the contour
  \(\gamma\) depicted in Figure \ref{fig:rect}, then letting
  \(R \to \infty\).
	\begin{figure}[h]
    \begin{center}
      \begin{tikzpicture}
        \draw [<->, ax] (-1,0) -- (1,0);
        \draw [->, ax] (0,0) -- (0,1);
        \draw [contour]
        (-1,0) coordinate [dot] node [below] {\(-R\)}
        -- coordinate [dot] node [below] {\(0\)}
        node [near end, below] {\(\gamma_1\)}
        (1,0) coordinate [dot] node [below] {\(R\)}
        -- node [right] {\(\gamma_2\)}
        (1,1) coordinate [dot] node [above right] {\(R+ib\)}
        -- node [near start, above] {\(\gamma_3\)}
        (-1,1) coordinate [dot] node [above left] {\(-R+ib\)}
        -- node [left] {\(\gamma_4\)} cycle;
      \end{tikzpicture}
    \end{center}
		\caption{The contour for exercise \ref{exerciseRectangle}.}
		\label{fig:rect}
	\end{figure}
	Be sure to justify your reasoning.

  \begin{solution}

  \end{solution}
\end{exercise}

\begin{exercise}\label{exerciseSemiCircle}
	Evaluate the improper integral
  \[
		\int_0^{\infty} \frac{1- \cos x}{x^2}\,dx
  \]
	by integrating the function \(f(z) = \frac{1-e^{iz}}{z^2}\) over the
  contour \(\gamma\) depicted in Figure \ref{fig:semicircle},
	\begin{figure}[h]
    \begin{center}
      \begin{tikzpicture}
        \tikzset{declare function={eps=1/6;}}
        \draw [<->, ax] (-1,0) -- (1,0);
        \draw [->, ax] (0,0) -- (0,1);
        \coordinate [dot] () at (0,0);
        \node [below] at (0,0) {\(0\)};
        \draw [contour]
        (eps,0) coordinate [dot] node [below] {\(\epsilon\)}
        -- (1,0) coordinate [dot] node [below] {\(R\)}
        arc [radius=1, start angle=0, end angle=180]
        node [near start, above right] {\(\gamma_R\)}
        node [midway, above right] {\(iR\)}
        coordinate [dot] node [below] {\(-R\)}
        -- (-eps,0) coordinate [dot] node [below] {\(-\epsilon\)}
        arc [radius=eps, start angle=180, end angle=0]
        node [near end, above right] {\(\gamma_\epsilon\)}
        -- cycle;
      \end{tikzpicture}
    \end{center}
		\caption{The contour for exercise \ref{exerciseSemiCircle}.}
		\label{fig:semicircle}
	\end{figure}
	then letting \(R \to \infty\) and \(\epsilon \to 0\). Be sure to
  justify your reasoning.

  \begin{solution}

  \end{solution}
\end{exercise}

\end{document}
