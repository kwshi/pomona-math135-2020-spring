\documentclass {../math135}

\title {Homework 10}
\author {}
\date {Tuesday 4/14}

\begin {document}

\begin {exercise}
	The \emph {convex hull} \(H\) of a collection \(z_1,z_2,\ldots,z_n\)
  of complex numbers is the smallest convex set containing each of the
  \(z_i\) (see Figure \ref {fig:convex-hull}).  In fact, \(H\) is the
  set of all \emph {convex combinations} of the \(z_i\):
  \[
		H = \left \{
      \sum_{i=1}^n a_i z_i
      : 0 \le a_i \le 1, \, \sum_{i=1}^n a_i = 1
    \right \}.
  \]
	Prove that if \(p(z)\) is a nonconstant polynomial of degree
  \(n \ge 1\), the zeros of \(p'(z)\) lie in the convex hull \(H\)
  formed by the zeros \(z_1, z_2, \dots, z_n\) of \(p(z)\).  This is
  the \emph {Gauss--Lucas theorem}.

  \begin {figure}[h]
    \begin {center}
      \includegraphics {fig-convex-hull.pdf}
      \caption{If you imagine the points \(z_1, z_2, \dots, z_n\) as
        ``poles'' sticking out of the complex plane and start with a
        giant rubber band which contains each of the \(z_i\) in its
        interior, then the convex hull is the area occupied by the
        rubber band and its interior once it is done contracting.}
      \label {fig:convex-hull}
    \end {center}
  \end {figure}

  \begin {solution}

  \end {solution}

\end {exercise}

\begin {exercise}
	Suppose that \(f \colon \Omega \to \CC\) is analytic and \(z_0\) is
  a zero of order \(n\) of \(f\).  Does there exist a branch of
  \(\sqrt[n] f\) on some neighborhood of \(z_0\)?

  \begin {solution}

  \end {solution}

\end {exercise}

\begin {exercise}
	Suppose \(f \colon \DD \to \DD\) is analytic.  What are the possible
  numbers of fixed points in \(\DD\) that \(f\) may have?

  \begin {solution}

  \end {solution}

\end {exercise}

\begin {exercise}
	Let \(\Omega = \DD \setminus \{0, \frac 1 2, \frac 1 3, \ldots\}\)
  and suppose \(f \colon \Omega \to \CC\) is bounded and analytic.
  Can \(f\) be extended to a bounded analytic function
  \(\tilde f \colon \DD \to \CC\)?

  \begin {solution}

  \end {solution}

\end {exercise}

\begin {exercise}
	Suppose that \(f\) is an entire function that is not a polynomial.
  Is \(f^{-1}(\{0\})\) necessarily nonempty?  Is it infinite?  What
  about \(f^{-1}(\{0, 1\})\)?

  \begin {solution}

  \end {solution}

\end {exercise}

\end {document}