\documentclass{../math135}

\title{Homework 6}
\author{}
\date{Tuesday 3/3}

\begin{document}

\begin{exercise}
	Depict \(i^i\) graphically.

  \begin{solution}
  \end{solution}

\end{exercise}

\begin{exercise}
	Find a formula for \(\cos^{-1} z\) (the inverse cosine) in terms of
  complex logarithms and roots.\footnote{Since similar reasoning
    applies to \(\sqrt z = \exp\prn*{\frac 1 2 \log z}\),
    \(\sin^{-1} z\), \(\tan^{-1} z\), etc., every function from
    calculus can be written in terms of the exponential function (if
    we think of \(\log z\) as the ``inverse function'' for \(e^z\)).}

  \begin{solution}
  \end{solution}

\end{exercise}

\begin{exercise}
	Show that there is a branch of the function
  \[
    f(z) = \sqrt{\frac{z+1}{z-1}}
  \]
  on \(\CC \setminus [-1, 1]\).  Sketch the corresponding Riemann
  surface.  How many sheets does it have?

  \begin{solution}
  \end{solution}

\end{exercise}

\begin{exercise}
	For \(n = 1, 2, \ldots\), derive the formula
  \[
		\frac{1}{2\pi} \int_0^{2\pi} \cos^{2n} t \dif t
    = \frac{1 \cdot 3 \cdot 5 \cdots (2n-1)}{2 \cdot 4 \cdot 6 \cdots (2n)}
  \]
	by writing \(\cos t\) in terms of the complex exponential function
  and interpreting the result as a contour integral over the unit
  circle \(\abs z = 1\).

  \begin{solution}
  \end{solution}

\end{exercise}

\end{document}